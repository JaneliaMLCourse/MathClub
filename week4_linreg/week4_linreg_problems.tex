\documentclass[11pt]{article}

\usepackage{graphicx}
\usepackage{amsmath}
\usepackage{amssymb}

\newcommand\xmtx{\boldsymbol{X}}
\newcommand\amtx{\boldsymbol{A}}

\newcommand\betavec{\boldsymbol{\beta}}
\newcommand\yvec{\boldsymbol{y}}
\newcommand\yhatvec{\boldsymbol{\hat{y}}}


\begin{document}

\section{ Derivatives }

\subsection{ Sum of squared differences }

Remember that the $x_i$ and $y_i$ are constant, and we're allowed to
change the parameters $a$ and $b$. If our cost function $C$ is the SSD,
then:

\begin{equation}
    C(a,b) = \sum_i (y_i - (ax_i + b))^2.
\end{equation}

Compute the derivatives of SSD with respect to $a$ and $b$.
\emph{ Hint: remember the chain rule if you don't want to multiply out
the square.}


\begin{equation}
        \frac{ dC }{ da } =  \\
\end{equation}
\begin{equation}
        \frac{ dC }{ db } =  \\
\end{equation}

%% SOLUTION
%\begin{equation}
%        \frac{ dC }{ da } =  \\
%\end{equation}
%\begin{equation}
%        \frac{ dC }{ db } =  \\
%\end{equation}


\section{Setting up problems}

\subsection{Write linear regression with matrices}

Write down this system of linear equations using matrix-vector
multiplication.  \emph{ Hint: Remember what size matrices are allowed
to be left-or-right multiplied with each other}

\begin{equation}
    \begin{array}{l}
    y_1 = \alpha_0 + \alpha_1 x_1  + \alpha_2 x_2 + \alpha_3 x_3 \\
    y_2 = \beta_0 + \beta_1 x_1  + \beta_2 x_2 + \beta_3 x_3 \\
    \end{array}
\end{equation}

\begin{equation}
    \begin{bmatrix}
        y_1 &  y_2 
    \end{bmatrix}
    = 
    \begin{bmatrix}
        1 & x_1 & x_2 & x_3
    \end{bmatrix}
    \begin{bmatrix}
        & & & &  \\ 
        & & ? & & \\
        & & & &   \\ 
    \end{bmatrix}
\end{equation}

%% SOLUTION
%\begin{equation}
%    \begin{bmatrix}
%        y_1 &  y_2 
%    \end{bmatrix}
%    = 
%    \begin{bmatrix}
%        1 & x_1 & x_2 & x_3
%    \end{bmatrix}
%    \begin{bmatrix}
%         \alpha_0 & \beta_0  \\ 
%         \alpha_1 & \beta_1  \\ 
%         \alpha_2 & \beta_2  \\ 
%         \alpha_3 & \beta_3  \\ 
%    \end{bmatrix}
%\end{equation}

\subsection{ Rewrite the same linear system }

Rewrite your answer above so that $y$ is a column vector.

\begin{equation}
    \begin{bmatrix}
        y_1 \\  y_2 
    \end{bmatrix}
    = 
    \begin{bmatrix}
         & &  \\ 
         & ? & \\
         & &  \\ 
    \end{bmatrix}
    \begin{bmatrix}
         & &  \\ 
         & ? & \\
         & &  \\ 
    \end{bmatrix}
\end{equation}



% SOLUTION
%\begin{equation}
%    \begin{bmatrix}
%        y_1 \\
%         y_2 
%    \end{bmatrix}
%    = 
%    \begin{bmatrix}
%         \alpha_0 & \alpha_1 & \alpha_2 & \alpha_3
%         \beta_0 & \beta_1 & \beta_2 & \beta_3 \\
%    \end{bmatrix}
%    \begin{bmatrix}
%        1   \\ 
%        x_1 \\
%        x_2 \\
%        x_3
%    \end{bmatrix}
%\end{equation}

\section { Normal Equations }

\subsection{ When do the Normal Equations not work ? }

\begin{enumerate}
    \item Write down a set of points $x_i, y_i$ such that we can't use
        the normal equations to solve for the best-fit line. \emph{ Hint: When
        will $X^TX$ not be invertible?} 
    \item What results would PCA give for the points you gave above?
\end{enumerate}
\section {Challenges}

\subsection{ Matrix "beta-squared" derivatives }

Here we'll confirm that $\frac{ \delta }{ \delta \betavec } ( \betavec^T \xmtx^T \xmtx \betavec )
    = 2 \xmtx^T \xmtx \betavec $

\begin{enumerate}
    \item Show that $\xmtx^T \xmtx$ is a symmetric matrix.  This means that if
        we rename it like this: $\amtx = \xmtx^T \xmtx$, then $a_{i,j} = a_{j,i}$
    \item Write $\betavec^T \amtx$ as a linear combination of the rows
        of $\amtx$.
    \item Write $\betavec^T \amtx \betavec$ as a linear combinations of
        dot products of rows of $\amtx$ and $\betavec$
    \item Expand out the dot products as sums
    \item Take the derivative with respect to one component of beta:
        $\beta_i$
    \item Use the propery that $\amtx$ is symmetric
    \item Concatenate into a vector, rearrange and QED! $\square$
\end{enumerate}



\end{document}
